\section*{Vorträge}
\subsection*{2022}
Anstehend: \enquote{Enactment of Futures: Crypto Twitter as Imaginary Practice}. Digital Futures in the Making: Imaginaries, Politics, and Materialities (Conference of the DGEKW Section “Digitisation in Everyday Life”), Hamburg (09/2022).\\[.25cm]Anstehend: \enquote{Prognose und Utopie: „Crypto Twitter“ als populärkultureller Finanz- und Zukunftsdiskurs}. Zukunftsentwürfe in der Populärkultur, Freiburg (6/2022).\\[.25cm]Anstehend: \enquote{Promise as Contingent and Anticipatory Practice: Tracing Futures in Multilateral Negotiations}. Making and Keeping Promises in Politics and Culture, Dresden (5/2022).\\[.25cm]Anstehend: \enquote{Utopian Investments: Notes on the Pragmatics of Crypto Twitter}. ELAN Workshop Series, Online (5/2022).\\[.25cm]\enquote{Zeitlichkeit und Aussensicht – Zeitebenen der Konstruktion Europas durch außereuropäische Akteure}. dgv-Kongress, Regensburg (4/2022).\\[.25cm]\enquote{Aesthetic Felicity: Narratological Approaches to the Aesthetics of Fact Checking, Counter-Narratives and Political Activism in Germany}. Populisms and their Aesthetics: Truthmaking, Faking, and the Politics of Affect in Audiovisual (Digital) Media, Heidelberg (3/2022).\\[.25cm]\enquote{Populäre Narrative des Politischen: Europaskeptizismus aus Sicht der Empirischen Kulturwissenschaft}. Antrittsvorlesung als Privatdozent an der Universität Zürich, Zürich (3/2022).\subsection*{2021}
\enquote{Roundtable Participation: Narratives of Global Cooperation}. Midterm Conference New Avenues of Global Cooperation Research, Centre for Global Cooperation Research, Duisburg (11/2021).\\[.25cm]mit Sarah May und Johannes Müske, \enquote{Morality as Organizational Practice (Introduction)}. Tagung Morality as Organizational Practice (dgv-Kommission Arbeitskulturen), Freiburg (11/2021).\\[.25cm]\enquote{Comment on Andrew Graan: Diplomatic Commentary and the Sovereignty Trap: Rethinking Political Reform Macedonia}. ELAN Workshop Series, Online (11/2021).\\[.25cm]\enquote{Kontext und politisches Erzählen (Keynote)}. Tagung Politisches Erzählen. Narrative, Genres, Strategien (dgv-Kommission Erzählforschung), Freiburg (08/2021).\\[.25cm]\enquote{Entangled Pasts, Dissonant Heritages: Heritage as Strategic Resource in EU External Relations}. SIEF-Kongress, Helsinki (6/2021).\\[.25cm]\enquote{Podium Contribution: Ethnologies of the Future: Anthropology and the EU-Agenda}. Civis-Workshop Studying European Culture and Society in Challenging Times, Tübingen (4/2021).\\[.25cm]\enquote{Selbstoptimierung und Navigational Capacity}. Tagung Optimierung des Selbst (Graduiertenkolleg Ethnographien des Selbst in der Gegenwart), Mainz (3/2021).\subsection*{2020}
\enquote{Populäre Narrative des Politischen: Europaskeptizismus aus Sicht der Empirischen Kulturwissenschaft}. Probevortrag im Rahmen des Habilitationsverfahrens zur Erlangung der Venia Legendi in Empirische Kulturwissenschaft, Universität Zürich, Zürich (12/2020).\\[.25cm]\enquote{Alltag als Konstante der Erfahrung?}. Workshop: Erfahrung: Konzeptionen und Standortbestimmungen eines Schlüsselbegriffs der Europäischen Ethnologie, Kiel (11/2020).\\[.25cm]\enquote{Of 'Truthiness' and 'Stickiness.' Narratological Approaches to Plausibility and Communicabilty in Digital Truth-Making}. Tagung Digital Truth-Making (dgv-Kommission Digitization in Everyday Life), Berlin (10/2020).\subsection*{2019}
\enquote{Durchschnitt als Trend: Orientierungen am Mittelmaß im Wohnen}. Tagung: Wohnen jenseits der Normen. Historische und aktuelle Perspektiven, Marburg (4/2019).\\[.25cm]\enquote{Comparison as Reflective and Affective Practice: Orientations Towards the Middle and Everyday Comparisons}. SIEF-Kongress, Santiago de Compostela (4/2019).\\[.25cm]\enquote{Akteure, Praktiken, Atmosphären: Elemente einer regionalen Konstellationsanalyse}. Bewerbungsvortrag auf die W3-Professur für Kulturanthropologie und Europäische Ethnologie mit Schwerpunkt in regionaler Kulturanalyse, Universität Freiburg, Freiburg (2/2019).\subsection*{2018}
\enquote{Vergleich als Wissen / Wissensproduktion im Vergleich}. Podiumsdiskussion Wissensalltag / Alltagswissen. Orte, Medien und Praktiken. Perspektiven aus Anthropologie und Kulturwissenschaft im Rahmen des Science Festivals 100 Ways of Thinking, Kunsthalle Zürich, Zürich (10/2018).\\[.25cm]\enquote{Antizipierender Vergleich. Zur zeitlichen Dimensionierung von Vergleichen am Beispiel von Handlungsorientierungen am Mittelmaß}. DGV-Hochschultagung Planen – Hoffen – Befürchten: Zukunft als Gegenstand und Herausforderung der Alltagskulturforschung, Bonn (09/2018).\\[.25cm]mit Sarah May, Johannes Müske, \enquote{Vernetzt, entgrenzt, prekär? Arbeit im Wandel und in gesellschaftlicher Diskussion. Eine Einführung}. Arbeitstagung der dgv-Kommission Arbeitskulturen, Zürich (09/2018).\\[.25cm]\enquote{Wissen, Ort, Vergleich. Praktiken des Wissens als Komparativ und Verortung}. Institut für Kulturanthropologie / Europäische Ethnologie, Universität Göttingen, Göttingen (6/2018).\\[.25cm]\enquote{Schlusskommentar}. Workshop des DFG-Projekts Partizipative Entwicklung ländlicher Regionen: Ländliche Alltagskulturen regieren. Perspektiven der Kulturanalyse politischer Prozesse in ländlichen Regionen, Bonn (6/2018).\\[.25cm]\enquote{Optimierung bis zum Mittelmaß? Sozialkomparative Orientierungen an der Mitte aus alltagskultureller Perspektive}. FRIAS Workshop Selbstoptimierung, Freiburg (6/2018).\\[.25cm]\enquote{Narratologisches Doppel: Text und Kontext in Online-Archiven und Interaktionssituationen}. Forschungsdesign 4.0. Datengenerierung und Wissenstransfer in interdisziplinärer Perspektive, Dresden (4/2018).\\[.25cm]\enquote{Angemessen, maßvoll, ausreichend: Handlungsorientierungen an der „Mitte“ und ihre alltagskulturellen Dimensionen}. Institut für Europäische Ethnologie, Universität Wien, Wien (1/2018).\subsection*{2017}
\enquote{Makro-Trends als Forschungsthema? Europäisch-ethnologische Themenbegrenzung am Beispiel der «Mitte»}. Tagung „Wie kann man nur dazu forschen?“ Themenpolitik in der Europäischen Ethnologie, Innsbruck (11/2017).\\[.25cm]mit Christian Ritter, \enquote{Zusammenarbeit(en). Praktiken der Koordination, Kooperation und Repräsentation in kollaborativen Prozessen. Eine Einführung}. Workshop Zusammenarbeit(en). Praktiken der Koordination, Kooperation und Repräsentation in kollaborativen Prozessen, Zürich (10/2017).\\[.25cm]\enquote{Zwischen Ermöglichung und Begrenzung: Zur subjektiven Plausibilisierung des Mittelmaßes als normative Orientierung}. Kongress der Deutschen Gesellschaft für Volkskunde (DGV), Marburg (09/2017).\\[.25cm]\enquote{Institution und Alltag: Kulturanalytische Zugänge zur Alltäglichkeit von Institutionen}. Institut für Sozialanthropologie und Empirische Kulturwissenschaft, Universität Zürich, Zürich (4/2017).\\[.25cm]\enquote{Of “Good Averages” and “Happy Mediums”: Normative Orientations towards an “Average” in Urban Housing}. SIEF-Kongress, Göttingen (3/2017).\\[.25cm]\enquote{Document Analysis as a Black Box: On the Contextualization of Speech Acts in Multilateral Negotiations}. Workshop Micro-Moves in International Institutions, Universität Potsdam, Potsdam (2/2017).\\[.25cm]mit Alejandro Esguerra, Katja Freistein, \enquote{Micro to Macro. On Generalizing from Communicative Approaches Towards International Institutions}. Workshop Micro-Moves in International Institutions, Universität Potsdam, Potsdam (2/2017).\subsection*{2016}
\enquote{Ambivalenz, Intention und Kompetenz zwischen Linguistischer Anthropologie und Narrationsanalyse}. Workshop Narrationsanalyse in der Europäischen Ethnologie, Universität Innsbruck, Innsbruck (09/2016).\\[.25cm]mit Regina F. Bendix, \enquote{,Kultur(-Erbe)‘ als flexibles Konzept in EU-Kulturpolitik und Außenbeziehungen}. Institut für Europäische Ethnologie, HU Berlin, Berlin (5/2016).\subsection*{2015}
\enquote{Welche Rolle spielt Gerechtigkeit in Verhandlungen um kulturelles Eigentum?}. DIES, Bonn (12/2015).\\[.25cm]mit Katja Freistein, Alejandro Esguerra, \enquote{Observing Micro-Practices, Making Generalizations: A Concept Note}. Workshop Studying Micro-Practices in (International) Institutions: Chances and Limitations of Theory-Building, Duisburg (11/2015).\\[.25cm]\enquote{Kein sichereres Mittel existirt zur Abwehr von allem Lupengesindel‘: Zur Technisierung und Legitimierung von Sicherheits- und Kontrollregimen um 1900}. Tagung „Der Alltag der (Un)Sicherheit. Ethnographisch-kulturwissenschaftliche Perspektiven auf die Sicherheitsgesellschaft“, Graz (11/2015).\\[.25cm]\enquote{Subjektiver Sinn, objektive Indikatoren? Zum Verhältnis von Wahrnehmung und Vermessung im freizeitsportlichen Rennradsport}. Kongress der Deutschen Gesellschaft für Volkskunde (DGV), Zürich (7/2015).\\[.25cm]\enquote{The Pragmatics of Multilateral Negotiations}. Midterm Conference Käte Hamburger Kolleg / Centre for Global Cooperation Research, Duisburg (7/2015).\\[.25cm]\enquote{Recognition and Multiculturalism: German Heritage Discourse in the European Context}. SIEF-Kongress, Zagreb (6/2015).\\[.25cm]\enquote{Culture Concepts and Normative Principles: On the Framing and Justification of Cultural Property in EU-Conventions}. Käte Hamburger Kolleg / Centre for Global Cooperation Research, Duisburg (5/2015).\\[.25cm]\enquote{Kultur als “Soft Power”? Zur Rahmung und Rechtfertigung von Kulturerbe in der Europäischen Union}. Abteilung Kulturanthropologie / Volkskunde, Bonn (5/2015).\subsection*{2014}
\enquote{Modalities of Normative Claims to Culture in Multilateral Negotiations}. Käte Hamburger Kolleg / Centre for Global Cooperation Research, Duisburg (11/2014).\\[.25cm]\enquote{Kooperation als Ressource: Zur Produktion kooperativen Alltagshandelns}. Tagung „Zum Umgang mit begrenzten Ressourcen”, Kiel (11/2014).\\[.25cm]\enquote{Producing Stability: On the Pragmatics of Multilateral Negotiations}. WISC, Frankfurt (08/2014).\\[.25cm]\enquote{Normative Forderungen über Kultur als Themenfeld transnationaler Kooperation}. Workshops "Mosaike der Legitimität", Konstanz (7/2014).\\[.25cm]\enquote{Welche Rolle spielt Gerechtigkeit in Diskussionen um kulturelles Eigentum?}. Vortragsreihe der Interdisziplinären DFG-Forschergruppe zu Cultural Property, Göttingen (1/2014).\subsection*{2013}
\enquote{Implicit Ethics: Normative Claims in International Negotiations on Traditional Knowledge}. Working Conference “Justice in Discourse”, Göttingen (4/2013).\\[.25cm]\enquote{Perspectives on Traditonal Knowledge: The Involvement of Indigenous and Local Communities in WIPO’s Committee on Intellectual Property and Traditional Knowledge}. SOGIP Seminar “Les questions de savoirs et de droit dans l’institutionnalisation internationale des autochtones”, Paris (1/2013).\subsection*{2012}
\enquote{Sprache und Anerkennung: Die Verortung von Subjekten in Diskursen}. Workshop Subjektbegriffe der Europäischen Ethnologie, Göttingen (12/2012).\\[.25cm]\enquote{Between Society and Culture: The Theory of Recogniton in Cultural Heritage Contexts}. Atelier de Recherche Trinational “Institutions, Territoires et Communautés: Perspectives sur le Patrimoine Culturel Immatériel Translocal, Villa Vigoni (10/2012).\\[.25cm]\enquote{The Indeterminacy of Cultural Heritage and Cultural Property in International Negotiations and Local Configurations}. Workshop Local Vocabularies of Heritage, Évora (2/2012).\subsection*{2011}
\enquote{Die Erfindung der Moral: Allmendgemeinschaften und Cultural Commons in der Diskussion um kulturelles Eigentum}. 38. Kongress der Deutschen Gesellschaft für Volkskunde (DGV), Tübingen (09/2011).\\[.25cm]mit Regina F. Bendix, \enquote{Speeding, Stalling, Editing: Maximal Communication for Minimal Results?}. Internationales Symposium der Interdisziplinären DFG-Forschergruppe zu Cultural Property, Göttingen (6/2011).\\[.25cm]\enquote{Scholarship and Policy – Oppositional Perspectives within Interdisciplinary Cooperation}. Internationales Symposium der Interdisziplinären DFG-Forschergruppe zu Cultural Property, Göttingen (6/2011).\subsection*{2010}
\enquote{Metapragmatics on the Global Stage: The Multiplicity of Meaning in International Negotiations}. Workshop zur Anthropology of International Institutions: How Ethnography Contributes to Understanding Mechanisms of Global Governance, Paris (6/2010).\\[.25cm]\enquote{Language Ideologies and International Institutions: The Case of the Intergovernmental Committee on Intellectual Property and Traditional Knowledge}. Department of Anthropology, UC Santa Cruz, Santa Cruz (4/2010).\\[.25cm]\enquote{Cultural Property als kulturanthropologisches Forschungsthema}. Institut für Kulturanthropologie / Europäische Ethnologie der Universität Göttingen, Göttingen (1/2010).\subsection*{2009}
\enquote{Ideologies in Motion: Language, Power and Performance in International Institutions}. Konferenz der American Anthropological Association (AAA), Philadelphia (12/2009).\\[.25cm]\enquote{On Tradition and the Politics of Difference}. International Symposium on Cultural Property, Interdisciplinary Research Unit on Cultural Property, Göttingen (11/2009).\\[.25cm]\enquote{Negotiating Tradition, Representing Social Structure: Culture, Community, and Language in WIPO’s Committee on Traditional Knowledge and Folklore}. International Summer University Local Knowledge and Open Borders: Creativity and Heritage, Tartu University, Tartu (7/2009).\\[.25cm]\enquote{Tradition und Folklore in der Weltorganisation für Geistiges Eigentum (WIPO) – Eine kommunikationsethnographische Perspektive}. Hamburger Gesellschaft für Volkskunde (HGV), Hamburg (5/2009).\subsection*{2008}
mit Regina F. Bendix, \enquote{Stalling and Speeding: Ways of Speaking at WIPO’s Intergovernmental Committee on Cultural Property}. Konferenz der European Association of Social Anthropologists (EASA), Ljubljana (08/2008).  
  
