\documentclass[11pt, a4paper]{article} % Document font size and paper size
%--------------------
\usepackage[ngerman]{babel}
\usepackage[T1]{fontenc}
\usepackage[utf8]{inputenc}
%--------------------
%\usepackage{fontspec} % Allows the use of OpenType fonts
\usepackage{microtype}
\usepackage[autostyle=true,german=quotes]{csquotes}
%--------------------
\clubpenalty = 10000 
\widowpenalty = 10000 
\displaywidowpenalty = 10000
%--------------------
\usepackage{geometry} % Allows the configuration of document margins
\geometry{a4paper, textwidth=5.5in, textheight=8.5in, marginparsep=7pt, marginparwidth=.6in} % Document margin settings
\setlength\parindent{0in} % Remove paragraph indentation
%--------------------
\usepackage{enumitem} 
\setitemize{leftmargin=*} 
%--------------------
\usepackage{lastpage}
%--------------------
\usepackage{fancyhdr}
\pagestyle{fancy} %eigener Seitenstil
\fancyhf{} %alle Kopf- und Fußzeilenfelder bereinigen
\fancyhead[L]{\textcolor{gray}{\scriptsize Curriculum Vitae\- •\- Version: \today\- •\- \href{http://github.com/sgroth/cv-data/}{http://github.com/sgroth/cv-data/}}} %Kopfzeile links
\renewcommand{\headrulewidth}{0pt} %obere Trennlinie
\fancyfoot[C]{\thepage\- von \pageref*{LastPage}} %Seitennummer
%--------------------
\usepackage[usenames,dvipsnames]{xcolor} % Custom colors
%--------------------
\usepackage{sectsty} % Allows changing the font options for sections in a document
%\usepackage[normalem]{ulem} % Custom underlining
%\usepackage{xunicode} % Allows generation of unicode characters from accented glyphs
%\defaultfontfeatures{Mapping=tex-text} % Converts LaTeX specials (``quotes'' --- dashes etc.) to unicode
%--------------------
\usepackage{marginnote} % For margin years
\newcommand{\years}[1]{\marginnote{\scriptsize #1}} % New command for including margin years
\renewcommand*{\raggedleftmarginnote}{}
\setlength{\marginparsep}{7pt} % Slightly increase the distance of the margin years from the contant
\reversemarginpar
%--------------------
\usepackage[xetex, bookmarks, colorlinks, breaklinks, pdftitle={CV Stefan Groth},pdfauthor={Stefan Groth}]{hyperref} % PDF setup - set your name and the title of the document to be incorporated into the final PDF file meta-information
\hypersetup{linkcolor=magenta,citecolor=magenta,filecolor=black,urlcolor=magenta} % Link colors
%--------------------
\usepackage{titlesec} 
\titlespacing*{\section}{0pt}{*5}{*1.2} 
%------------------------------------------------
\sectionfont{\rmfamily\mdseries\large\uppercase} % Set font options for sections
\subsectionfont{\rmfamily\mdseries\scshape\normalsize} % Set font options for subsections
\subsubsectionfont{\rmfamily\bfseries\upshape\normalsize} % Set font options for subsubsections
%--------------------
\usepackage[osf]{ebgaramond} % Garamond for rm
\usepackage[osf]{sourcesanspro} % SourceSansPro for ss
\usepackage[scale=0.9]{sourcecodepro} % SourceCodePro for tt
%--------------------
% Version Control
%\usepackage{gitfile-info}
%------------------------------------------------
\begin{document}
%------------------------------------------------
{\LARGE Dr. Stefan Groth}\\[1cm] % Your name
ISEK – Institut für Sozialanthropologie und Empirische Kulturwissenschaft\\
Populäre Kulturen\\ 
Universität Zürich\\
Affolternstrasse 56\\  
CH-8050 Zürich\\[.2cm]  
Telefon: +41 44 634 92 96\\ % Your phone number
E-Mail: \href{mailto:stefan.groth@uzh.ch}{stefan.groth@uzh.ch}\\
Homepage: \href{https://www.stefangroth.com}{https://www.stefangroth.com}\\\\
%--------------------
%--------------------
\section*{Derzeitige Position}
\years{seit 09/2016}\emph{Oberassistent / Leitung Labor Populäre Kulturen}, ISEK – Institut für Sozialanthropologie und Empirische Kulturwissenschaft, Populäre Kulturen, Universität Zürich % Your current or previous employment position
%------------------------------------------------
\section*{Forschungsschwerpunkte}
    Normative Dimensionen der Alltagskultur; 
    Sprachanalytische Ansätze; 
    Politische Anthropologie und Organisationsforschung; 
    Kulturerbe und kulturelles Eigentum; 
    Kulturwissenschaftliche Sportforschung.
%------------------------------------------------
\section*{Frühere Stellen}
\years{2015--2016}Post-Doc, Kulturanthropologie / Volkskunde, Institut für Archäologie und Kulturanthropologie, Universität Bonn. \\
\years{2014--2015}Post-Doctoral Fellow, \href{https://www.gcr21.org}{Käte Hamburger Kolleg / Centre for Global Cooperation Research}, Universität Duisburg–Essen. Projekt: Culture as Resource and Diplomacy: Between Geopolitics and Issues-Based Policy.\\
\years{2011--2014}Post-Doc, Institut für Kulturanthropologie / Europäische Ethnologie, Georg-August-Universität Göttingen. Projekt: The Ethics of/in Negotiating and Regulating Cultural Property (Teilprojekt der Interdisziplinären DFG-Forschergruppe \href{http://www.cultural-property.org}{Die Konstituierung von Cultural Property: Akteure, Diskurse, Kontexte, Regeln}).\\
\years{2008--2011}Wissenschaftlicher Mitarbeiter / Doktorand, Institut für Kulturanthropologie / Europäische Ethnologie, Georg-August-Universität Göttingen. Projekt: Kommunikationsmuster und Entscheidungsfindung über cultural property im internationalen Gremium World Intellectual Property Organization (Teilprojekt der Interdisziplinären DFG-Forschergruppe \href{http://www.cultural-property.org}{Die Konstituierung von Cultural Property: Akteure, Diskurse, Kontexte, Regeln}).\\
%------------------------------------------------
\section*{Ausbildung}
\years{2008--2011}Dr. phil. in Kulturanthropologie / Europäische Ethnologie, Georg-August-Universität Göttingen (DE). Doktorarbeit: \enquote{Negotiating Tradition: The Pragmatics of International Deliberations on Cultural Property} (summa cum laude). Gutachter: Prof. Dr. Regina F. Bendix (Göttingen), Prof. Dr. Donald F. Brenneis (UC Santa Cruz).\\
\years{10/03--03/08}M.A. in Soziologie, Fächer: Soziologie, Kulturanthropologie / Europäische Ethnologie, Wirtschafts- und Sozialpsychologie, Georg-August-Universität Göttingen (DE). Magisterarbeit: \enquote{Entwicklung von Open-Source-Software: Soziologische Diskussion einer spezifischen Form von Innovationsnetzwerk} (Sehr gut).\\
\years{10/06--01/07}Erasmus-Programm, Fächer: Public Relations, Università degli Studi di Udine (IT). 
%------------------------------------------------
\section*{Aufenthalte im Ausland}
\years{03/10--04/10}DAAD-Kurzstipendium für Doktoranden, Forschungsaufenthalte an der University of Chicago, University of California at Santa Cruz, School for Advanced Research in Santa Fe, USA.\\
\years{2012}Erasmus Teaching Staff Mobility Grant, Seminar für Kulturwissenschaft und Europäische Ethnologie, Universität Basel, Schweiz.\\
\years{10/06--01/07}Erasmus-Programm, Public Relations, Università degli Studi di Udine, Italien.
%------------------------------------------------
\section*{Weiterbildungen und Qualifikationen}
\years{05/2017}Leadership Skills for Postdocs, Universität Zürich.\\
\years{01/2017}Project Management for Successful Postdocs, Universität Zürich.\\
\years{10/2013}Workshop Gendersensible Didaktik, Universität Göttingen.\\
\years{2013--2014}Zertifikatsprogramm Hochschuldidaktik, Georg-August-Universität Göttingen. Kurse:\\[.2cm] 
Prüfen und Prüfungsrecht (01/2013); 
Aktivierende Methoden (03/2013); 
Basiskompetenzen Hochschullehre 1 (03/2013); 
Präsentieren und Rhetorik (05/2013); 
Kollegiale Lehrhospitation (06/2013); 
Basiskompetenzen Hochschullehre 2 (07/2013); 
Einsatz von E-Learning-Tools in der Lehre (11/2013); 
Kollegiales Praxisgespräch (11/2013); 
Beratung von Studierenden (01/2014).
%------------------------------------------------
\section*{Professionelle Aktivitäten und Mitgliedschaften}
\years{seit 2017}Delegierter des Mittelbaus in der Kommission für Forschung und Nachwuchsförderung, Philosophische Fakultät der Universität Zürich\\
\years{seit 2014}\href{http://www.konkurrenz.uni-kiel.de/de}{DFG-Netzwerk „Wettbewerb und Konkurrenz: Zur kulturellen Logik kompetitiver Figurationen“}\\
\years{seit 2013}\href{https://networks.h-net.org/h-folk}{H-Folk-Netzwerk, Editor}\\
\years{seit 2008}\href{http://easaonline.org/networks/lawnet/index.shtml}{EASA Network “Anthropology of Law and Rights” (LAWNET)}\\
\years{2010--2014}Zentrum für Theorie und Methodik der Kulturwissenschaften (ZTMK), Göttingen, Vorstand\\
\years{2015--2016}Vorstand Institut für Archäologie und Kulturanthropologie, Universität Bonn (Gruppe der akademischen Mitarbeiter, Stellv.)\\
Schweizerische Gesellschaft für Volkskunde, Sektion Zürich\\
Deutsche Gesellschaft für Volkskunde (DGV)\\
European Association of Social Anthropologists (EASA)\\
Societé Internationale d´Ethnologie et de Folklore (SIEF)\\
Rheinische Vereinigung für Volkskunde (RVV)%--------------------
\subsection*{Gutachtertätigkeiten}
International Journal of Heritage Studies%------------------------------------------------
\section*{Organisatorische Aktivitäten}
    Organisation mit Dr. des. Christian Ritter (Collegium Helveticum, Zürich), Workshop Zusammenarbeit(en). Praktiken der Koordination, Kooperation und Repräsentation in kollaborativen Prozessen, 5.-6. Oktober 2017, Labor Populäre Kulturen, Institut für Sozialanthropologie und Empirische Kulturwissenschaft, Universität Zürich / Collegium Helveticum, Zürich (\href{http://www.stefangroth.com/assets/pdf/CfP_Zusammenarbeiten_Zürich.pdf}{Call for Papers als PDF}).\\[.25cm]
    In-House Workshop Perspektiven ethnographischer Kulturanalyse, 4.-5. Mai 2017, Labor Populäre Kulturen, Institut für Sozialanthropologie und Empirische Kulturwissenschaft, Universität Zürich.\\[.25cm]
    Organisation mit Dr. Alejandro Esguerra (Potsdam) und Dr. Katja Freistein (Duisburg), Internationaler Workshop Micro-Moves in International Institutions, Standing Group Sociology of International Relations (AK SiB) / Deutsche Vereinigung für Politische Wissenschaft, February 9-10, 2017, Universität Potsdam (\href{https://stefangroth.com/img/micromoves-program.pdf}{Programm als PDF}). Keynotes von Karin Knorr-Cetina (Chicago) und Thomas Scheffer (Frankfurt).\\[.25cm]
    Organisation mit Dr. Katja Freistein (Duisburg) und Dr. Alejandro Esguerra Portocarrero (Duisburg), Interdisziplinärer Workshop Studying Micro-Practices in (International) Institutions: Chances and Limitations of Theory-Building, November 26-27, 2015, Centre for Global Cooperation Research (GCR), Duisburg (\href{https://stefangroth.com/img/micro-programm.pdf}{Programm als PDF}).\\[.25cm]
    Organisation mit Prof. Dr. Charles Briggs (UC Berkeley) und Prof. Dr. Regina Bendix, International Working Conference Justice in Discourse, April 4–5, 2013, Göttingen (\href{https://stefangroth.com/img/justice-programm.pdf}{Programm als PDF}). Mit Beiträgen von Prof. Dr. Srikant Sarangi (Cardiff), Prof. Dr. Jan Blommaert (Tilburg), Prof. Dr. Patrick Eisenlohr (Göttingen), Prof. Dr. Charles Briggs (Berkeley), Prof. Dr. Regina Bendix (Göttingen), Dr. Stefan Groth (Göttingen), Ruth Goldstein (Berkeley), Ina Lehmann (Bremen). (\href{http://www.hsozkult.de/conferencereport/id/tagungsberichte-4961}{Tagungsbericht auf H-Soz-Kult})\\[.25cm]
    Organisation mit Nadine Wagener-Böck M.A., Workshop „Subjektbegriffe der Europäischen Ethnologie“ (“Concepts of the ‘Subject’ in European Ethnology”), December 13-14, 2012, Göttingen (\href{https://stefangroth.com/img/subjekt-programm.pdf}{Programm als PDF}). Mit Beiträgen von Prof. em. Dr. Johannes Fabian (Amsterdam), Prof. Dr. Andreas Schmidt (Kiel), PD Dr. Jochen Bonz (Bremen), Christine Öldorp M.A. (Zürich), Dr. Gerrit Herlyn (Hamburg), Julia Butschatskaja M.A. (Sankt Petersburg), Nadine Heymann M.A. (Berlin), Dr. Thomas Dörfler (Lüneburg), Martina Röthl M.A. (Innsbruck), Erdem Evren M.A. (Berlin), Maria Schwertl M.A. (Göttingen).
%--------------------
\section*{Lehrerfahrungen}
\emph{Institution und Alltag: Ethnographische Zugänge zu Institutionen} (Master), FS 2017. Populäre Kulturen, ISEK - Institut für Sozialanthropologie und Empirische Kulturwissenschaft, Universität Zürich.\\[.25cm]
\emph{Alltagskultur und Normativität} (Bachelor), FS 2017. Populäre Kulturen, ISEK - Institut für Sozialanthropologie und Empirische Kulturwissenschaft, Universität Zürich.\\[.25cm]
\emph{Empirische Kulturanalyse des Alltags, Methodenseminar} (Master), SS 2016. Abteilung Kulturanthropologie / Volkskunde, Institut für Archäologie und Kulturanthropologie, Friedrich-Wilhelms-University Bonn.\\[.25cm]
\emph{Projektseminar} (Master), WS 2015. Abteilung Kulturanthropologie / Volkskunde, Institut für Archäologie und Kulturanthropologie, Friedrich-Wilhelms-University Bonn.\\[.25cm]
\emph{Projektseminar} (Master), SS 2016. Abteilung Kulturanthropologie / Volkskunde, Institut für Archäologie und Kulturanthropologie, Friedrich-Wilhelms-University Bonn.\\[.25cm]
\emph{Kulturerbe und kulturelles Eigentum} (Master), WS 2015. Abteilung Kulturanthropologie / Volkskunde, Institut für Archäologie und Kulturanthropologie, Friedrich-Wilhelms-University Bonn.\\[.25cm]
\emph{Kultur und Kommunikation: Einführung in die Ethnographie der Kommunikation und linguistische Anthropologie} (Master), WS 2013. Institut für Kulturanthropologie / Europäische Ethnologie, Georg-August-Universität Göttingen.\\[.25cm]
\emph{Vorlesung Kulturtheorien (Regina Bendix, Carola Lipp), Übernahme der Vorlesung „Strukturalismus“}, SS 2013. Institut für Kulturanthropologie / Europäische Ethnologie, Georg-August-Universität Göttingen.\\[.25cm]
\emph{Kulturerbe verhandeln: Methodische und theoretische Zugänge}, HS 2012. Seminar für Kulturwissenschaft und Europäische Ethnologie, Universität Basel.\\[.25cm]
\emph{Vorlesung Kulturtheorien (Regina Bendix, Carola Lipp), Übernahme der Vorlesung „Frühe Paradigmen der Kulturanthropologie/Europäischen Ethnologie: Evolutionismus, Kulturrelativismus, Funktionalismus“}, SS 2012. Institut für Kulturanthropologie / Europäische Ethnologie, Georg-August-Universität Göttingen.\\[.25cm]
\emph{Kulturtheorien}, SS 2011. Institut für Kulturanthropologie / Europäische Ethnologie, Georg-August-Universität Göttingen.\\[.25cm]
\emph{Das Konzept des Kulturrelativismus}, WS 2009. Institut für Kulturanthropologie / Europäische Ethnologie, Georg-August-Universität Göttingen. 
%--------------------
\section*{Vorträge}
\subsection*{2017}
\enquote{Makro-Trends als Forschungsthema? Europäisch-ethnologische Themenbegrenzung am Beispiel der «Mitte»}. Tagung „Wie kann man nur dazu forschen?“ Themenpolitik in der Europäischen Ethnologie, Innsbruck (11/2017).\\[.25cm]
\enquote{Zwischen Ermöglichung und Begrenzung: Zur subjektiven Plausibilisierung des Mittelmaßes als normative Orientierung}. Kongress der Deutschen Gesellschaft für Volkskunde (DGV), Marburg (09/2017).\\[.25cm]
\enquote{Institution und Alltag: Kulturanalytische Zugänge zur Alltäglichkeit von Institutionen}. Institut für Sozialanthropologie und Empirische Kulturwissenschaft, Universität Zürich, Zürich (4/2017).\\[.25cm]
\enquote{Of “Good Averages” and “Happy Mediums”: Normative Orientations towards an “Average” in Urban Housing}. SIEF-Kongress, Göttingen (3/2017).\\[.25cm]
\enquote{Document Analysis as a Black Box: On the Contextualization of Speech Acts in Multilateral Negotiations}. Workshop Micro-Moves in International Institutions, Universität Potsdam, Potsdam (2/2017).\\[.25cm]
mit Alejandro Esguerra, Katja Freistein, \enquote{Micro to Macro. On Generalizing from Communicative Approaches Towards International Institutions}. Workshop Micro-Moves in International Institutions, Universität Potsdam, Potsdam (2/2017).
\subsection*{2016}
\enquote{Ambivalenz, Intention und Kompetenz zwischen Linguistischer Anthropologie und Narrationsanalyse}. Workshop Narrationsanalyse in der Europäischen Ethnologie, Universität Innsbruck, Innsbruck (09/2016).\\[.25cm]
mit Regina F. Bendix, \enquote{,Kultur(-Erbe)‘ als flexibles Konzept in EU-Kulturpolitik und Außenbeziehungen}. Institut für Europäische Ethnologie, HU Berlin, Berlin (5/2016).
\subsection*{2015}
\enquote{Welche Rolle spielt Gerechtigkeit in Verhandlungen um kulturelles Eigentum?}. DIES, Bonn (12/2015).\\[.25cm]
mit Katja Freistein, Alejandro Esguerra, \enquote{Observing Micro-Practices, Making Generalizations: A Concept Note}. Workshop Studying Micro-Practices in (International) Institutions: Chances and Limitations of Theory-Building, Duisburg (11/2015).\\[.25cm]
\enquote{Kein sichereres Mittel existirt zur Abwehr von allem Lupengesindel‘: Zur Technisierung und Legitimierung von Sicherheits- und Kontrollregimen um 1900}. Tagung „Der Alltag der (Un)Sicherheit. Ethnographisch-kulturwissenschaftliche Perspektiven auf die Sicherheitsgesellschaft“, Graz (11/2015).\\[.25cm]
\enquote{Subjektiver Sinn, objektive Indikatoren? Zum Verhältnis von Wahrnehmung und Vermessung im freizeitsportlichen Rennradsport}. Kongress der Deutschen Gesellschaft für Volkskunde (DGV), Zürich (7/2015).\\[.25cm]
\enquote{The Pragmatics of Multilateral Negotiations}. Midterm Conference Käte Hamburger Kolleg / Centre for Global Cooperation Research, Duisburg (7/2015).\\[.25cm]
\enquote{Recognition and Multiculturalism: German Heritage Discourse in the European Context}. SIEF-Kongress, Zagreb (6/2015).\\[.25cm]
\enquote{Culture Concepts and Normative Principles: On the Framing and Justification of Cultural Property in EU-Conventions}. Käte Hamburger Kolleg / Centre for Global Cooperation Research, Duisburg (5/2015).\\[.25cm]
\enquote{Kultur als “Soft Power”? Zur Rahmung und Rechtfertigung von Kulturerbe in der Europäischen Union}. Abteilung Kulturanthropologie / Volkskunde, Bonn (5/2015).
\subsection*{2014}
\enquote{Modalities of Normative Claims to Culture in Multilateral Negotiations}. Käte Hamburger Kolleg / Centre for Global Cooperation Research, Duisburg (11/2014).\\[.25cm]
\enquote{Kooperation als Ressource: Zur Produktion kooperativen Alltagshandelns}. Tagung „Zum Umgang mit begrenzten Ressourcen”, Kiel (11/2014).\\[.25cm]
\enquote{Producing Stability: On the Pragmatics of Multilateral Negotiations}. WISC, Frankfurt (08/2014).\\[.25cm]
\enquote{Normative Forderungen über Kultur als Themenfeld transnationaler Kooperation}. Workshops "Mosaike der Legitimität", Konstanz (7/2014).\\[.25cm]
\enquote{Welche Rolle spielt Gerechtigkeit in Diskussionen um kulturelles Eigentum?}. Vortragsreihe der Interdisziplinären DFG-Forschergruppe zu Cultural Property, Göttingen (1/2014).
\subsection*{2013}
\enquote{Implicit Ethics: Normative Claims in International Negotiations on Traditional Knowledge}. Working Conference “Justice in Discourse”, Göttingen (4/2013).\\[.25cm]
\enquote{Perspectives on Traditonal Knowledge: The Involvement of Indigenous and Local Communities in WIPO’s Committee on Intellectual Property and Traditional Knowledge}. SOGIP Seminar “Les questions de savoirs et de droit dans l’institutionnalisation internationale des autochtones”, Paris (1/2013).
\subsection*{2012}
\enquote{Sprache und Anerkennung: Die Verortung von Subjekten in Diskursen}. Workshop Subjektbegriffe der Europäischen Ethnologie, Göttingen (12/2012).\\[.25cm]
\enquote{Between Society and Culture: The Theory of Recogniton in Cultural Heritage Contexts}. Atelier de Recherche Trinational “Institutions, Territoires et Communautés: Perspectives sur le Patrimoine Culturel Immatériel Translocal, Villa Vigoni (10/2012).\\[.25cm]
\enquote{The Indeterminacy of Cultural Heritage and Cultural Property in International Negotiations and Local Configurations}. Workshop Local Vocabularies of Heritage, Évora (2/2012).
\subsection*{2011}
\enquote{Die Erfindung der Moral: Allmendgemeinschaften und Cultural Commons in der Diskussion um kulturelles Eigentum}. 38. Kongress der Deutschen Gesellschaft für Volkskunde (DGV), Tübingen (09/2011).\\[.25cm]
mit Regina F. Bendix, \enquote{Speeding, Stalling, Editing: Maximal Communication for Minimal Results?}. Internationales Symposium der Interdisziplinären DFG-Forschergruppe zu Cultural Property, Göttingen (6/2011).\\[.25cm]
\enquote{Scholarship and Policy – Oppositional Perspectives within Interdisciplinary Cooperation}. Internationales Symposium der Interdisziplinären DFG-Forschergruppe zu Cultural Property, Göttingen (6/2011).
\subsection*{2010}
\enquote{Metapragmatics on the Global Stage: The Multiplicity of Meaning in International Negotiations}. Workshop zur Anthropology of International Institutions: How Ethnography Contributes to Understanding Mechanisms of Global Governance, Paris (6/2010).\\[.25cm]
\enquote{Language Ideologies and International Institutions: The Case of the Intergovernmental Committee on Intellectual Property and Traditional Knowledge}. Department of Anthropology, UC Santa Cruz, Santa Cruz (4/2010).\\[.25cm]
\enquote{Cultural Property als kulturanthropologisches Forschungsthema}. Institut für Kulturanthropologie / Europäische Ethnologie der Universität Göttingen, Göttingen (1/2010).
\subsection*{2009}
\enquote{Ideologies in Motion: Language, Power and Performance in International Institutions}. Konferenz der American Anthropological Association (AAA), Philadelphia (12/2009).\\[.25cm]
\enquote{On Tradition and the Politics of Difference}. International Symposium on Cultural Property, Interdisciplinary Research Unit on Cultural Property, Göttingen (11/2009).\\[.25cm]
\enquote{Negotiating Tradition, Representing Social Structure: Culture, Community, and Language in WIPO’s Committee on Traditional Knowledge and Folklore}. International Summer University “Local Knowledge and Open Borders: Creativity and Heritage”, Tartu University, Tartu (7/2009).\\[.25cm]
\enquote{Tradition und Folklore in der Weltorganisation für Geistiges Eigentum (WIPO) – Eine kommunikationsethnographische Perspektive}. Hamburger Gesellschaft für Volkskunde (HGV), Hamburg (5/2009).
\subsection*{2008}
mit Regina F. Bendix, \enquote{Stalling \& Speeding: Ways of Speaking at WIPO’s Intergovernmental Committee on Cultural Property}. Konferenz der European Association of Social Anthropologists (EASA), Ljubljana (08/2008).
%--------------------
%--------------------
\vfill{} % Whitespace before final footer
%--------------------
\begin{center}
{\scriptsize Version: \today\- •\- \href{http://github.com/sgroth/cv-data/}{http://github.com/sgroth/cv-data/}} 
\end{center}
%--------------------
\end{document}
